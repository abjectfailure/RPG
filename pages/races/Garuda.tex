\chapter*{Garuda}
\addcontentsline{toc}{section}{\indent Garuda}

\subsection*{Background}

Garuda are avian creatures that live in deserts. Though descended from a flighted species, Garudas evolved towards moving quickly and gracefully across large sand dunes and eventually lost their wings. This is a small source of shame for the race.

Garudas travel the deserts following changing wind streams that will create or destroy an oasis. However, due to the scarcity of resources many other creatures have the same pattern, especially predators who will wait around an oasis for an easy meal. The most deadly of these are the Naga, giant snakes who feed upon the nomadic Garuda.

\subsection*{Physical Description}

Garudas are moderately sized, roughly comparable to the Makara. Both stand at the same height of four to five feet, however the Garuda are a bit lighter at only 75 to 120 pounds. This is due to the porous nature of their bones, a remnant from their evolutionary ancestors.

A Garuda has a light coating of down covering their body leading to a plumage extending from their neck and the back of their head. This helps regulate their body temperature through the excessive heat during the day and the plunging temperatures at night in the desert. At birth, this down can be tinged with a variety of colors, which then become bleached white in the sun as the Garuda ages.

\subsection*{Personality}

group-focused, conservative, mostly insular but not isolationist

\newpage

\begin{figure}[ht!]
	\includegraphics[width=\linewidth]{img/Garuda.png}
\end{figure}

\subsection*{Racial Traits}
\begin{monsterbox}{Garuda}
	\vspace{.1in}
	\hline
	\stats[
	STR = -1,
	DEX = +1,
	VIT = -1,
	FOC = +1,
	WILL = +1
	]
	\hline
\end{monsterbox}