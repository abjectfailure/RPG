\chapter*{Makara}
\addcontentsline{toc}{section}{\indent Makara}

\subsection*{Background}

Makara are another ancient race, coexisting with the Umibozu society while other races were in their infancy. Though still an air-breathing race, Makara will often build their towns in sunken caverns, however they have no qualms about building cities on open land, unlike their aquatic contemporaries.

Another difference between the two oldest races comes in their culture. Whereas the Umibozu prefer seclusion and developed an intricate culture, the Makara have a much simpler culture and routinely get involved in the affairs of other races. Primarily, this involvement comes in the form of war, either trying to gain territory from other groups, or fighting alongside others to gain spoils of war and for training.

\subsection*{Physical Description}

Makara are thin scaled bipeds, with a coloring that varies from a sickly yellow color to a deep forest green. They stand smaller than an average Human at around four  to five feet tall. They are also substantially lighter, weighing only 90 to 160 pounds. A Makara has claws, a forked tongue, and sharp teeth as any other lizard would, however their bodies no longer sport a tail as their ancestors once did. Male Makara’s vertebrae protrude down their back in a set of spikes and are used as status symbols 

Though omnivorous, Makara prefer to eat fresh meat and will form hunting parties to gather it even when ample plants are available for food. Hunting parties can be composed of both male and female Makara as there is much less patriarchy among Makara than other races.

\subsection*{Personality}

Makara have a very utilitarian sense of ethics, primarily the ends always justify the means and might makes right to them. A Makara adventurer may push weaker party members around until they prove their worth to them. However, unlike some of the other races, this denigration is only done because the Makara does not want to get killed in battle and doesn’t necessarily see the other member as a “lesser species”.

Makara are also quite stubborn, once they have a way of doing things that works, it is difficult to get them to change. This has been a major component in the stability of their society over the eons, however also contributes to their relative lack of advancement over time.

\newpage

\begin{figure}[ht!]
	\includegraphics[width=\linewidth]{img/Makara.png}
\end{figure}

\subsection*{Racial Traits}
\begin{monsterbox}{Makara}
	\vspace{.1in}
	\hline
	\stats[
	STR = +1,
	DEX = -1,
	VIT = +1,
	FOC = -1,
	WILL = +1
	]
	\hline
\end{monsterbox}