\chapter*{Class Name}
\addcontentsline{toc}{section}{\indent Class Name}

\subsection*{Description}

\begin{commentbox}{}
	\lipsum[1]
\end{commentbox}

\subsection*{Class Features}

% For more columns, you can say \begin{dndtable}[your options here}.
% For instance, if you wanted three columns, you could say
% \begin{dndtable}{XXX}. The usual host of tabular parameters are
% aailable as well.
\header{Class Table}
\renewcommand{\arraystretch}{1.25}
\begin{dndtable}[p{1.25cm} p{1.5cm} X]
	\textbf{Level}  & \textbf{Stats} & \textbf{Features}\\
	1st  && Talent \\
	2nd  && Ability 1 \\
	3rd  && Ability 2 \\
	4th  & +1 Any & Class Unlock \\
	5th  && Talent \\
	6th  && Ability 1 Upgrade \\
	7th  && Ability 2 Upgrade \\
	8th  & +1 Any & Class Unlock \\
	9th  && Talent  \\
	10th && Capstone
\end{dndtable}

\subsubsection{Talent}
Your initial training results in a minor talent. Choose one of the following options. You cannot choose the same talent more than once, even if you later get to choose again.

\subsubsection{Talent 1}
Description of Talent 1

\subsubsection{Talent 2}
Description of Talent 2

\subsubsection{Talent 3}
Description of Talent 3

\vspace{.1 in}

\subsection*{Ability 1}
\lipsum[1]

\vspace{.1 in}

\subsection*{Ability 2}
\lipsum[1]

\vspace{.1 in}

\subsection*{Class Unlock}
\lipsum[1]

\vspace{.1 in}

\subsection*{Talent}
Your training results in a minor talent. Choose one of the following options:

\subsubsection{Talent 1}
Description of Talent 1

\subsubsection{Talent 2}
Description of Talent 2

\subsubsection{Talent 3}
Description of Talent 3

\vspace{.1 in}

\subsection*{Improved Ability 1}
\lipsum[1]

\vspace{.1 in}

\subsection*{Improved Ability 2}
\lipsum[1]

\vspace{.1 in}

\subsection*{Talent}
Your training results in a minor talent. Choose one of the following options:

\subsubsection{Talent 1}
Description of Talent 1

\subsubsection{Talent 2}
Description of Talent 2

\subsubsection{Talent 3}
Description of Talent 3

\vspace{.1 in}

\subsection*{Capstone}
\lipsum[1]
