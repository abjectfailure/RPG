\chapter*{Adventurer}
\addcontentsline{toc}{section}{\indent Adventurer}


\section*{Description}

\begin{commentbox}{}
	\lipsum[1]
\end{commentbox}


% For more columns, you can say \begin{dndtable}[your options here}.
% For instance, if you wanted three columns, you could say
% \begin{dndtable}{XXX}. The usual host of tabular parameters are
% aailable as well.
\header{The Adventurer}
\renewcommand{\arraystretch}{1.25}
\begin{dndtable}[p{1.25cm} p{1.5cm} X]
	\textbf{Level}  & \textbf{Stats} & \textbf{Features}\\
	1st  && Talent \\
	2nd  && Adventurer's Path \\
	3rd  && Menace \\
	4th  & +1 Any & Class Unlock \\
	5th  && Talent \\
	6th  && Adventurer's Path Ability \\
	7th  && Improved Menace \\
	8th  & +1 Any & Class Unlock \\
	9th  && Talent  \\
	10th && Capstone
\end{dndtable}

\subsection*{Talent}
Your initial training results in a minor talent. Choose one of the following options. You cannot choose the same talent more than once, even if you later get to choose again.

\subsubsection{Weapon Focus}
You gain a bonus to damage with a weapon of your choice.

\subsubsection{Catch Off Guard}
You have advantage on attacks against enemies that are adjacent to one of your allies.

\subsubsection{Second Wind}
Once per encounter, upon taking damage, you may use your reaction to regain hit points.

\vspace{.1 in}

\subsection*{Adventurer's Path}
At second level, you begin to be able to keep special tabs on a specific opponent in combat. As a bonus action, you can choose to mark an opponent visible to you. The mark lasts until you can no longer see the target, the target becomes incapacitated, or you use this ability again to mark a different creature. You may only have one creature marked with this ability at a time, if you use this ability to mark a second creature, the first mark ends. 

Choose one of the following paths below to determine what you do with this mark:

\subsubsection{Knight's Challenge}
If a creature marked by you makes a melee attack against an adjacent ally, you may use your reaction to make an attack against that creature.

\subsubsection{Hunter's Quarry}
Whenever you make a successful attack against a creature you have advantage against and that you have marked, you can choose to deal extra damage against that creature. You may deal this extra damage only once per round.

\subsubsection{Third Mark Type}
Description

\vspace{.1 in}

\subsection*{Menace}
Your presence and gaze tend to evoke certain emotions from your marked prey. Choose one of the following penalties that apply to all foes marked by you:

\subsubsection{Vexation}
Marked creatures take a -1 penalty to attacks made against any creature other than you.

\subsubsection{Trepidation}
Whenever you successfully attack a marked creature, their movement speed is halved until the beginning of your next turn.

\subsubsection{Terror}
Whenever a marked creature ends it's movement farther away from you than it started, you gain advantage on your next attack made against that creature.

\vspace{.1 in}

\subsection*{Class Unlock}
\lipsum[1]

\vspace{.1 in}

\subsection*{Talent}
Your training results in a minor talent. Choose one of the following options:

\subsubsection{Keen Edge}
Your attacks score a critical hit on a roll of 19 or 20.

\subsubsection{Talent 2}
Description of Talent 2

\subsubsection{Talent 3}
Description of Talent 3

\vspace{.1 in}

\subsection*{Improved Menace}
Your presence and gaze tend to evoke certain emotions from your marked prey. Choose one of the following penalties that apply to all foes marked by you:

\subsubsection{Enervation}
Whenever you successfully attack a marked creature, they must roll twice for any damage rolls they make and use the lower of the two results. This lasts until the end of their next turn.

\subsubsection{Weaken}
Increase the die size of any dice you roll for damage against a marked creature.

\subsubsection{Infuriate}
Marked creatures take a -1 penalty to attacks made against any creature other than you. This penalty stacks with the Vexation Menace.

\vspace{.1 in}

\subsection*{Talent}
Your training results in a minor talent. Choose one of the following options:

\subsubsection{Talent 1}
Description of Talent 1

\subsubsection{Talent 2}
Description of Talent 2

\subsubsection{Talent 3}
Description of Talent 3

\vspace{.1 in}

\subsection*{Capstone}
\lipsum[1]